%TeX

\section{Reversing the MBR}

\subsection{RE400 what?}

\begin{frame}{RE400 what?\ldots}
    \begin{columns}
        \column{0.4\textwidth}
	        \begin{itemize}
                \item Challenge worth 400 points
                \item Reverse Engineering category
                \item We get some hints right away\ldots
                \begin{itemize}
                	\item This is an MBR
                	\item \ldots from an x86 system
        	    \end{itemize}
            \end{itemize}
        \column{0.6\textwidth}
            \Graphic{re400}
    \end{columns}
\end{frame}

\begin{frame}{A place to start\ldots}
    \begin{columns}
        \column{0.5\textwidth}
	        \begin{itemize}
                \item<1-> Wikipedia, of course!
		        \begin{itemize}
    	            \item<2-> 512 bytes
        	        \item<2-> MBR signature: 55 AA
            	    \item<2-> "expected to contain real mode machine language instructions"
                	\item<2-> little-endian
                	\item<2-> loads at 0000:7C00
	            \end{itemize}
            \end{itemize}
        \column{0.5\textwidth}
            \Graphic{wikimbr}
    \end{columns}
\end{frame}

\subsection{Tools}

\begin{frame}[t]{Tool Time!}
    \framesubtitle{qemu (gift wrapped)}
    \begin{columns}
        \begin{column}[t]{0.3\textwidth}
            \begin{itemize}
                \item<1-> \texttt{-s} (gdb stub)
                \item<1-> \texttt{-S} (suspend)
                \item<1-> \texttt{-vnc:1} (enable remote VNC monitor)
                \item<2-> QEMU/Monitor
                \begin{itemize}
                    \item<2-> \texttt{info registers}
                    \item<2-> \texttt{system reset}
                \end{itemize}
            \end{itemize}
        \end{column}
        \begin{column}[t]{0.7\textwidth}
            \only<1>{
                \Graphic{launch-qemu} \newline \newline
            	\Graphic{qemu-vnc}
			}
            \only<2>{\Graphic{qemu-info-registers-1}}
        \end{column}
    \end{columns}
\end{frame}

\begin{frame}[t]{Tool Time!}
    \framesubtitle{gdb}
    \begin{columns}
        \begin{column}[t]{0.5\textwidth}
            \begin{itemize}
                \item<1-> \texttt{target remote localhost:1234}
                \item<1-> \texttt{set architecture i8086} \\
                    (bootloaders are 16 bit, right?)
                \item<1-> \texttt{display/i \$pc} - print program counter
                \item<1-> \texttt{br *0xADDR} - set breakpoint
                \item<1-> \texttt{si} - run one instruction
                \item<1-> \texttt{c} - continue
                \item<2-> \texttt{info reg} and \texttt{info frame}
                \item<3-|@uncover> \texttt{x /CT 0xADDR} - display C units of type T
                \item<3-|@uncover> \texttt{set \{int\}0xADDR} = 42
                \item<3-|@uncover> \texttt{set \{char[4]\} 0xADDR} = "AAA"
            \end{itemize}
        \end{column}
        \begin{column}[t]{0.5\textwidth}
            \only<1>{
                \Graphic{launch-gdb} \newline \newline
                \Graphic{gdb-target}
            }
            \only<2>{\Graphic{gdb-info-reg-1}}
            \only<3>{\Graphic{gdb-info-frame}}
            \only<4-5>{
                \uncover<4->{\Graphic{gdb-x-set-1} \newline \newline}
                \uncover<5->{\Graphic{gdb-x-set-2}}
            }
            \only<6>{
                \Graphic{after-set}
            }
        \end{column}
	\end{columns}
\end{frame}

\begin{frame}[t]{Tool Time!}
    \framesubtitle{IDA}
    \begin{columns}
        \begin{column}[t]{0.4\textwidth}
            \begin{itemize}
                \item Rebase
                \only<2->{
                	\item Common ASM
                	\only<2>{
         		       	\begin{itemize}
                            \item \texttt{int}
                            \item \texttt{mov}
                            \item \texttt{inc}, \texttt{dec}
                            \item \texttt{or}, \href{https://twitter.com/andnxor}{\texttt{and}, \texttt{not}, \texttt{xor}}\ldots
                            \item \texttt{cmp}
                            \item \texttt{jmp}, \texttt{jz}, \texttt{jge}, \texttt{jle}\ldots
	                	\end{itemize}
	                }
	            }
	            \only<3->{
	            	\item Registers
	            	\only<3>{
         		       	\begin{itemize}
                            \item \texttt{E AX} - Accumulator
                            \item \texttt{E CX} - Counter
                            \item \texttt{E DX} - Data
                            \item \texttt{E BX} - Base
                            \item \texttt{E SP} - Stack pointer
                            \item \texttt{E BP} - Stack base pointer
                            \item \texttt{E SI} - Source
                            \item \texttt{E DI} - Destination
	                	\end{itemize}
	                }
	            }
			\end{itemize}
        \end{column}
        \begin{column}[t]{0.6\textwidth}
			\only<1>{
				\Graphic{rebase}
			}
			\uncover<2->{
				\Graphic{ida-first}
			}
        \end{column}
	\end{columns}
\end{frame}

\subsection{Disassembly and reversing}

\begin{frame}{Disassembly\ldots}
    \begin{columns}
        \column{0.3\textwidth}
            \begin{itemize}
                \item Look for hints
                \begin{itemize}
                	\uncover<2->{
         	 	      	\item Loops
	     	         }
	          	    \uncover<3->{
	            		\item Comparisons
	            	}
	            	\uncover<4->{
	            		\item Unknowns
		            }
				\end{itemize}
			\end{itemize}
		\column{0.7\textwidth}
			\uncover<2->{
				\Graphic{loop}
			}
			\uncover<3->{
				\Graphic{cmp1}
			}
			\uncover<4->{
				\Graphic{sse}
			}
	\end{columns}
\end{frame}

\begin{frame}{Putting it all together\ldots}
    \begin{columns}
        \column{0.4\textwidth}
            \begin{itemize}
                \item We know that 0x7C6F compares user input to "flag"
                \uncover<2->{\item There's a bunch of complicated instructions right after that cmp}
	     	    \uncover<3->{\item We can use gdb and qemu/monitor to see what's happening...}
	     	    \uncover<4->{\item Now we just have to work backwards from there.}
			\end{itemize}
		\column{0.6\textwidth}
			\only<1>{
				\Graphic{cmp1}
			}
			\only<2>{
				\Graphic{sse}
			}
			\only<3->{
				\Graphic{reg1}
			}
			\uncover<4>{
				\Graphic{reg2}
			}
	\end{columns}
\end{frame}

\begin{frame}{Other useful tools\ldots that I didn't know about}
	\begin{itemize}
		\item Binary Ninja
 		\item pwndbg
 		\item Radare
 	\end{itemize}
\end{frame}
